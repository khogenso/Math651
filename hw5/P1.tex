\documentclass{homework651}

\usepackage[all,cmtip]{xy}

% The following commands set up the material that appears
% in the header.
\doclabel{Math 651: Homework 5, Problem 1 Redo}
\docauthor{Kirk Hogenson}
\docdate{}

\input{hwextras.tex}
\newcommand\nextprob{\newpage}
\newcommand\ra{\rightarrow}
\newcommand\calB{\mathcal{B}}
\newcommand\calS{\mathcal{S}}

\begin{document}
\begin{aproblems}

\hproblem Prove the following variation of Exercise 3.32.
\begin{subproblems}
\item  The space $(X_1\times X_2)\times X_3$ is homeomorphic to
$X_1\times X_2 \times X_3$.  You may not use the words ``open'' or ``closed''
at any point in your proof.  (\emph{Hint}: Use the Characteristic Property, Luke!)

\item A projection map from an arbitrary product space is an open map.

\item An arbitrary product of Hausdorff spaces is Hausdorff.

\item A countable product of second countable spaces is second countable.

\end{subproblems}

\subsol
Let $Y = (X_1\times X_2)\times X_3$ and $Z = X_1\times X_2 \times X_3$
and define $f:Y\ra Z$ as $f((x_1,x_2),x_3)=(x_1,x_2,x_3)$.  Evidently
$f$ is a bijection.

Observe that these diagrams commute:

\centerline{
\xymatrix{
Y \ar[r]^f \ar[d]_{\pi_{1}} & Z \ar[d]^{\pi_{i}} \\
X_1\times X_2 \ar[r]^{\pi_i} & X_i }
\qquad\qquad\qquad
\xymatrix{
Y \ar[r]^f \ar[dr]_{\pi_3\circ f} & Z \ar[d]^{\pi_3} \\
 & X_3 }}

The first diagram implies that $\pi_i\circ\pi_1=f\circ\pi_1$.  Since
a composition of projection functions is continuous, by the characteristic
property of the product topology $f$ is continuous in its first component.
The second diagram shows that $f$ is continuous in its
second component, so $f$ is continuous.

Now $f^{-1}:Z\ra Y$ is $f^{-1}(x_1,x_2,x_3)=((x_1,x_2),x_3)$, and
we have these diagrams that commute:

\centerline{
\xymatrix{
X_1\times X_2\times X_3 \ar[r]^-{f^{-1}} \ar[ddr]_{\pi_{i}} & (X_1\times X_2)\times X_3 \ar[d]^{\pi_{1}} \\
 & X_1\times X_2 \ar[d]_{\pi_i} \\
 & X_i }
\qquad\qquad
\xymatrix{
Z \ar[r]^{f^{-1}} \ar[dr]_{\pi_2\circ f^{-1}} & Y \ar[d]^{\pi_2} \\
 & X_3 }
}

From the left diagram,
we have $\pi_i \circ \pi_1 \circ f^{-1} = \pi_i$, and since $\pi_i\circ\pi_1$
and $\pi_i$ are continuous, $f^{-1}$ is, too, in its first two components.
From the right diagram, the diagonal is $\pi_2\circ f^{-1}=\pi_3$, which is
continuous, so $f^{-1}$ is continuous.

Hence $f$ is a homeomorphism, therefore $Y$ and $Z$ are homeomorphic.

\newpage

\subsol
Let $\pi: \prod_{\alpha\in I} X_\alpha \ra X_0$ be a projection map from
the arbitrary product space $\prod_{\alpha\in I} X_\alpha$, where $I$ is an index
collection, to $X_0\in\{X_\alpha\}_{\alpha\in I}$.

Let $U$ be a basic open set in $\prod_{\alpha\in I} X_\alpha$,
so $U=\prod_{\alpha\in I} U_\alpha$,
where $U_\alpha$ is open in $X_\alpha$, and all but finitely many
are $X_\alpha$.  Then, $\pi(U)=U_0$ where $U_0$
is open in $X_0$.  Thus $\pi$ is an open map.

\subsol
Let $\prod_{\alpha\in I} X_\alpha$ be an arbitrary product space,
where $X_\alpha$ is
Hausdorff for each $\alpha\in I$.  Let $x,y\in\prod_{\alpha\in I} X_\alpha$
with $x\ne y$.
Then for some $\beta\in I$, we will have $x_\beta\ne y_\beta$ where
$x_\beta,y_\beta\in X_\beta\in\{X_\alpha\}_{\alpha\in I}$.

The topological space $X_\beta$ is Hausdorff so there exist sets $U_1$ and
$U_2$ open in $X_\beta$ with $U_1\cap U_2=\emptyset$.

Let $\pi_\beta:\prod_{\alpha\in I}\ra X_\beta$ be the projection function
to $X_\beta$ and define 
$V_1= \pi_\beta^{-1}(U_1)$,
and
$V_2=\pi_\beta^{-1}(U_2)$.
These are open sets
in $\prod_{\alpha\in I} X_\alpha$ and $V_1\cap V_2=\emptyset$.  Further,
$x\in V_1$ and $y\in V_2$, so $\prod_{\alpha\in X_\alpha}$ is Hausdorff.

\vspace{.3in}

\subsol
Let $X=\prod_{n=1}^{\infty} X_n$ be a product of countably many spaces
$X_n$, where each $X_n$ is second countable.  Hence for each $X_n$ we have
a countable basis $\calB_n=\{B_k\}_{k=1}^{\infty}$.

Define $\calS=\{\pi_n^{-1}(B_k):B_k\in\calB_n \text{ and } k,n\in\Nats\}$,
where $\pi_n:X\ra X_n$ is the projection function onto the $n$-th component.
Note that $\calS$ is countable.

Now take any $x=(x_1,x_2,...)\in X$.  Then $x_1\in B_k$
for some $B_k\in\calB_1$.  So, $\pi_1^{-1}(B_k)=B_k\times\prod_{n=2}^{\infty} X_n$,
so $x\in \pi_1^{-1}(B_k)\in\calS$.  Hence $\calS$ covers $X$ and so it
is a sub-basis for $X$.

We can then form a basis for $X$ from $\calS$ as follows:
$$ \calB = \left\{ \bigcap_{k=1}^{N} S_k : S_k\in\calS\right\}.$$
Observe that $\calB$ is countable, as it is composed of all finite intersections
of elements of the countable set $\calS$.

\end{aproblems}
\end{document}

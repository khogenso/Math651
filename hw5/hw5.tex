\documentclass{homework651}

\usepackage[all,cmtip]{xy}

% The following commands set up the material that appears
% in the header.
\doclabel{Math 651: Homework 5}
\docauthor{Kirk Hogenson}
\docdate{18 February 2015}

\input{hwextras.tex}
\newcommand\nextprob{\newpage}
\newcommand\ra{\rightarrow}
\newcommand\calB{\mathcal{B}}

\begin{document}
\begin{aproblems}

\hproblem Prove the following variation of Exercise 3.32.
\begin{subproblems}
\item  The space $(X_1\times X_2)\times X_3$ is homeomorphic to
$X_1\times X_2 \times X_3$.  You may not use the words ``open'' or ``closed''
at any point in your proof.  (\emph{Hint}: Use the Characteristic Property, Luke!)

\item A projection map from an arbitrary product space is an open map.

\item An arbitrary product of Hausdorff spaces is Hausdorff.

\item A countable product of second countable spaces is second countable.

\end{subproblems}

\subsol
Let $Y = (X_1\times X_2)\times X_3$ and $Z = X_1\times X_2 \times X_3$
and define $f:Y\ra Z$ as $f((x_1,x_2),x_3)=(x_1,x_2,x_3)$.  Evidently
$f$ is a bijection.

Consider the diagrams:

\centerline{
\xymatrix{
Y \ar[r]^f \ar[dr]_{\pi_{12}\circ f} & Z \ar[d]^{\pi_{12}} \\
 & X_1\times X_2 }
\qquad
\xymatrix{
Y \ar[r]^f \ar[dr]_{\pi_3\circ f} & Z \ar[d]^{\pi_3} \\
 & X_3 }
}

In each, we see that the diagonal map is merely the projection function
from $Y$ onto its first and second components, and so is continuous.  Hence
by the characteristic property of the product topology, $f$ must be continuous
as well.

Now $f^{-1}:Y\ra Z$ is $f^{-1}(x_1,x_2,x_3)=((x_1,x_2),x_3)$, and
we have similar diagrams:

\centerline{
\xymatrix{
Z \ar[r]^{f^{-1}} \ar[dr]_{\pi_{1}\circ f^{-1}} & Y \ar[d]^{\pi_{1}} \\
 & X_1\times X_2 }
\qquad
\xymatrix{
Z \ar[r]^{f^{-1}} \ar[dr]_{\pi_3\circ f^{-1}} & Y \ar[d]^{\pi_3} \\
 & X_3 }
}

Again we have that the diagonal maps are projections of $Z$ onto its
first two components, and its third component.  Since this is continuous,
$f^{-1}$ is continuous and hence $f$ is a homeomorphism.  Therefore
$Y$ and $Z$ are homeomorphic.

\vspace{.3in}

\subsol
Let $\pi: \prod_{\alpha\in I} X_\alpha \ra X_0$ be a projection map from
the arbitrary product space $\prod_{\alpha\in I} X_\alpha$, where $I$ is an index
collection, to $X_0\in\{X_\alpha\}_{\alpha\in I}$.

Let $U$ be a basic open set in $\prod_{\alpha\in I} X_\alpha$,
so $U=\prod_{\alpha\in I} U_\alpha$,
where $U_\alpha$ is open in $X_\alpha$.  Then, $\pi(U)=U_0$ where $U_0$
is open in $X_0$.  Thus $\pi$ is an open map.

\newpage

\subsol
Let $\prod_{\alpha\in I} X_\alpha$ be an arbitrary product space,
where $X_\alpha$ is
Hausdorff for each $\alpha\in I$.  Let $x,y\in\prod_{\alpha\in I} X_\alpha$
with $x\ne y$.
Then for some $\beta\in I$, we will have $x_\beta\ne y_\beta$ where
$x_\beta,y_\beta\in X_\beta\in\{X_\alpha\}_{\alpha\in I}$.

The topological space $X_\beta$ is Hausdorff so there exist sets $U_1$ and
$U_2$ open in $X_\beta$ with $U_1\cap U_2=\emptyset$.

Define $V_1=\prod_{\alpha\in I\setminus\beta} X_\alpha \times U_1$, and
$V_2=\prod_{\alpha\in I\setminus\beta} X_\alpha \times U_2$.  These are open sets
in $\prod_{\alpha\in I} X_\alpha$ and $V_1\cap V_2=\emptyset$.  Further,
$x\in V_1$ and $y\in V_2$, so $\prod_{\alpha\in X_\alpha}$ is Hausdorff.

\vspace{.3in}

\subsol
Let $X=\prod_{n=1}^{\infty} X_n$ be a product of countably many spaces
$X_n$, where each $X_n$ is second countable.  Hence for each $X_n$ we have
a countable basis $\calB_n$.

Define $\calB=\{\prod_{n=1}^{\infty} B_n : B_n\in\calB_n \text{ for each } n\}$.

Observe that $\calB$ is countable, and that each $B\in\calB$ is open in $X$
because each $B_n$ is open in $X_n$, and that $\calB$ covers $X$.
Further, notice that
$$ B=\prod_{n=1}^{\infty} B_n \cap \prod_{n=1}^{\infty} \hat B_n =
\prod_{n=1}^{\infty} B_n\cap \hat B_n. $$
Each $B_n\cap\hat B_n\in\calB_n$, so $B\in\calB$, thus $\calB$ is
closed under finite intersection.  Hence $\calB$ is a basis for $X$.  Since
it is countable, $X$ is second countable.

\nextprob
\hproblem Problem 3-6\\
Let $X$ be a topological space.  The \emph{diagonal} of $X\times X$ is the subset
$\Delta=\{(x,x):x\in X\}\subseteq X\times X$.  Show that $X$ is Hausdorff
if and only if $\Delta$ is closed in $X\times X$.

\solution
Suppose $X$ is Hausdorff.  Consider
$\Delta^c=\{(x,y):x,y\in X \text{ and } x\ne y\}\subseteq X\times X$.
Let $(x,y)\in\Delta^c$, where $x,y\in X$.  Therefore $x\ne y$.
Because $X$ is Hausdorff, there exist open sets $U_x$ and $U_y$ in $X$
with $x\in U_x$ and $y\in U_y$ and $U_x\cap U_y=\emptyset$.  Now define
$U_p\subseteq X\times X$ as
$U_p=U_x\times U_y=\{(\hat x,\hat y):\hat x\in U_x\text{ and }
\hat y\in U_y\}$.

Note that $U_p$ is open in $X\times X$
since $X\times X$ has the product topology.
Now, take any $(\hat x,\hat y)\in U_p$,
and observe that $\hat x\ne\hat y$
because $\hat x\in U_x\subseteq X$ and $\hat y\in U_y\subseteq X$,
and $U_x\cap U_y=\emptyset$.
Hence $(\hat x,\hat y)\in\Delta^c$, which means $U_p\subseteq\Delta^c$.

Therefore $U_p$ is an open neighborhood of $(x,y)$ in $\Delta^c$ with
$U_p\subseteq\Delta^c$, so $\Delta^c$ is open in $X\times X$
and hence $\Delta$ is closed in $X\times X$.

Conversely, suppose $\Delta$ is closed in $X\times X$, and take any $x,y\in X$
with $x\ne y$.  Thus $(x,y)\in\Delta^c$. Since $\Delta^c$ is open in
$X\times X$, we can find a basic open set $B$ with $(x,y)\in B$
such that $B\subseteq\Delta^c$.

Every basic open set in the product topology can be expressed as
$B=U\times V$ with $U$ and $V$ open in $X$,
so let $U$ and $V$ be
such sets. Then $x\in U$ and $y\in V$, and since
$U\times V\cap\Delta=\emptyset$ we must have $x'\ne y'$ for
any $(x',y')\in U\times V$, which implies that
$U\cap V=\emptyset$;
hence $X$ is Hausdorff.

\nextprob
\hproblem Problem 3-7\\
Show that the space $X$ of Problem 2-22 is homeomorphic to
$\mathbb{R}_d\times\mathbb{R}$, where $\mathbb{R}_d$ is the set
$\mathbb{R}$ with the discrete topology.

Problem 2-22 defined $X$ as follows: for any fixed $a,b,c\in\Reals$,
let $I_{abc}$ be the subset of $\Reals^2$ defined by
$I_{abc} = \{(c,y):a<y<b\}$.  Let $\mathcal{B}$ be the collection of
all nonempty subsets of $\Reals^2$ of the form $I_{abc}$ for
$a,b,c\in\Reals$.  Let $X=\Reals^2$ as a set, but with the topology
generated by $\mathcal{B}$.

\textbf{Lemma:} \textit{The set $\mathcal{B}$ is a basis for a topology.}
(This is part (a) of Problem 2-22.)\\
Take any point $(x,y)\in\Reals^2$.  Define $a=y-1$ and $b=y+1$, then
$(x,y)\in I_{abx}$.  Thus $\mathcal{B}$ covers $\Reals^2$.

To show that $\mathcal{B}$ satisfies the refinement property, we show it
is closed under nonempty intersection.  Take
$B_1=I_{abc}$ and $B_2=I_{def}$.  For a nonempty intersection we must
have $c=f$ and that the open intervals $(a,b)$ and $(d,e)$ overlap in $\Reals$.
Then
$B_1\cap B_2=I_{abc}\cap I_{def}=
  \{(c,y):\max(a,d)<y<\min(b,e)\}\in\mathcal{B}$.\hfill$\Box$

\solution
Define $f:X\rightarrow \Reals_d\times\Reals$ as $(x,y)\rightarrow(x,y)$.
I.e., $f$ is the identity map.  Observe that $f$ is a bijection.

Let $\mathcal{B}_d = \{\{x\}:x\in\Reals\}$.  We showed in class that this
was a basis for the discrete topology on $\Reals$.  Let $\mathcal{B}_i$
be the set of open intervals in $\Reals$, which is a basis for $\Reals$.
Hence sets in $\Reals_d\times\Reals$ of the form
$B_d\times B_i$ with $B_d\in\mathcal{B}_d$ and $B_i\in\mathcal{B}_i$,
form a basis for the product topology on $\Reals_d\times\Reals$.

Let $U=B_d\times B_i=\{c\}\times(a,b)$ be a basic open set in
$\Reals_d\times\Reals$.  Then $f^{-1}(U)=I_{abc}$, which is open in $X$,
hence $f$ is continuous.

Now consider $f^{-1}:\Reals_d\times\Reals\rightarrow X$.  Let $U$ be a basic
open set in $X$, i.e., $U=I_{abc}$.  The preimage of $U$ under $f^{-1}$ is
$\{c\}\times(a,b)$ which is open in $\Reals_d\times\Reals$, hence $f^{-1}$
is continuous.

Thus $f$ is a homeomorphism, and therefore $X$ is homeomorphic to $\Reals_d\times\Reals$.

\nextprob
\hproblem Problem 3-8\\
Let $X$ denote the Cartesian product of countably infinitely many copies of
$\mathbb{R}$ (which is just the set of all infinite sequences of real numbers),
endowed with the box topology.  Define a map $f:\mathbb{R}\rightarrow X$
by $f(x)=(x,x,x,...)$.  Show that $f$ is not continuous, even though each of
its component functions is.

\solution
Consider the set $U_n\subseteq \RR$ defined by $U_n=(-1/n,1/n)$ for $n\in\Nats$.
Observe that $U_n$ is open in $\RR$.
Also, note that $0\in U_n$ for all values of $n$.

But if $x>0$, then there is an $n\in\Nats$ for which $1/n<x$, hence
$x\not\in U_n$ for some $n\in\Nats$.  Similarly, if $x<0$, there is
an $n\in\Nats$ for which $x<-1/n$, so $x\not\in U_n$ for some $n\in\Nats$.
So, the only $x\in\RR$ with $x\in U_n$ for all $n\in\Nats$ is $x=0$.

Define $U=\prod_{n=1}^{\infty} U_n$.  Then under the box topology, $U$ is open
in $X$.  But we just saw that $f^{-1}(U) = \{0\}$ which is closed in $\RR$.

Hence $f$ is not continuous.

\nextprob
\hproblem Problem 3-9\\
Let $X$ be as in the preceding problem.  Let $X^+\subseteq X$ be the subset
consisting of sequences of strictly positive real numbers, and let $z$ denote
the zero sequence, that is, the one whose terms are $z_i=0$ for all $i$.
Show that $z$ is in the closure of $X^+$, but there is no sequence of elements
of $X^+$ converging to $z$.  Then use the sequence lemma to conclude that $X$
is not first countable, and thus not metrizable.

(A part of the)\\
\textbf{Sequence Lemma}:
\textit{If $X$ is a first countable space, then for any nonempty $A\subseteq X$, and
$x\in\overline{A}$, $x$ is a limit of a sequence of
points in $A$.}\\
Let $U$ be an open set containing $x$.
Since $X$ is first countable, there is a countable
collection $\{B_n\}_{n=1}^{\infty}$ of nested neighborhoods of $x$, inside $U$.
Each $B_n$ is an open set containing $x$, therefore it will intersect $A$
because $x\in\overline{A}$.
Let $x_n$ be any point that is in $B_n\cap A$.  Thus $x_n$ is
a sequence that converges to $x$, because if $n>N$, then $B_n\subseteq B_N$.
Moreover, $x_n\in A$ for any $n$, as required.
\hfill$\Box$

\solution
Let $z\in U$ where $U$ is a basic open set in $X$. Thus $U$ can be
expressed as $U=\prod_{n=1}^{\infty} U_n$ where each $U_n$ is an open
subset of $\Reals$.  Since $z\in U$ we know $0\in U_n\subseteq\Reals$
for every $n$.

Let $\calB$ be the basis for $\Reals$ where every $B\in\calB$ in an
open interval.

Each $U_n$ is open in $\Reals$ so there exists a basic open set $B_n\in\calB$
where $0\in B_n\subseteq U_n$.  Since $B_n=(a_n,b_n)$, we must have
$0< b_n$.  So, we may choose a number $c_n$ such that $0< c_n< b_n$.
Then $c_n\in B_n\subseteq U_n$, and moreover $c_n\in\Reals^+$.

Hence $(c_1,c_2,...)\in X^+$, and thus $U$ contains points from $X^+$,
so $z\in \overline{X^+}$.

\vspace{.2in}

Let $\underline{x}_k$ be any sequence in $X^+$.

Define $U=\prod_{n=1}^{\infty} U_n$ where $U_n=(-x_n^{(n)}, x_n^{(n)})$.
Here $x_k^{(m)}$ refers to the $m$-th element in $\RR$ of the $k$-th
element in $X^+$ of the sequence $\underline{x}_k$.  Note that for each $n\in\Nats$,
we have $0\in U_n$, thus $z\in U$.

Now observe that we cannot find a $K\in\Nats$ for which
$k\ge K$ implies $\underline{x}_k \in U$, because by construction
$x_K^{(K)}\not\in U_K=(-x_K^{(K)},x_K^{(K)})$, which means
$\underline{x}_k\not\in U$ for all $k$.  Hence $\underline{x}_k$
does not converge to $z$.

\vspace{.2in}

Therefore, by the Sequence Lemma, $X$ must not be first countable, and therefore
not metrizable.
 
\nextprob
\hproblem Problem 3-10\\
Prove Theorem 3.41 (the characteristic property of disjoint union spaces): Suppose
that $(X_\alpha)_{\alpha\in A}$ is an indexed family of topological spaces, and
$Y$ is any topological space.  A map
$f:\coprod_{\alpha\in A} X_\alpha \rightarrow Y$ is continuous if and only if
its restriction to each $X_\alpha$ is continuous.  The disjoint union topology
is the unique topology on $\coprod_{\alpha\in A}X_\alpha$ with this property.

\solution

\centerline{\xymatrix{
\coprod_{\alpha\in A} X_\alpha \ar[dr]^f & X_\alpha \ar[l]_-{i_\alpha} \ar[d]^{f_\alpha} \\
 & Y}}

Let $f_\alpha=f|_{X_\alpha}$, and suppose $f_\alpha$ is continuous. The projection
map $\pi_\alpha$ is continuous, so $f=f_\alpha\circ\pi_\alpha$ is a composition
of continuous functions and is thus continuous.

Conversely, suppose $f$ is continuous and let $U$ be an open set in $Y$.
Then $f^{-1}(U)$ is open in $\coprod_{\alpha\in A} X_\alpha$.  By definition,
if a set is open in the disjoint union topology, then when the set is intersected
with each $X_\alpha$ the result is open in $X_\alpha$.  Hence
$f^{-1}(U)\cap X_\alpha$ is open in $X_\alpha$ for any $\alpha\in A$.

But $f^{-1}(U)\cap X_\alpha = f_\alpha^{-1}(U)$, so $f_\alpha^{-1}(U)$ is
open in $X_\alpha$ and hence $f_\alpha$ is continuous.

\vspace{.2in}

Next, suppose we have two topologies $\tau_1$ and $\tau_2$ on
$\coprod_{\alpha\in A} X_\alpha$ both satisfying the characteristic property.
Let $X_i$ be  $\coprod_{\alpha\in A} X_\alpha$ with the topology $\tau_i$,
for $i=1,2$, and let $id^{12}$ be the map identity map between $X_1$ and $X_2$,
and consider the diagram:

\centerline{\xymatrix{
X_1 \ar[dr]^{id^{12}} & X_\alpha \ar[l]_-{i_\alpha} \ar[d]^{id^{12}_\alpha} \\
 & X_2}}

The map $id^{12}_\alpha$ continuous by definition: if a set $U$ is open in $X_2$,
then $U\cap X_\alpha$ is open in $X_\alpha$.

So $id^{12}$ is continuous
by the characteristic property, which is satisfied by assumption.
Similarly, the identity map $id^{21}$ between $X_2$ and $X_1$ is continuous
as well.  Since $id^{12}$ is a bijection and $id^{12}=(id^{21})^{-1}$, $X_1$
and $X_2$ are homeomorphic and so $\tau_1=\tau_2$.

Thus there can be at most one disjoint union topology satisfying the characteristic
property.  We know there is at least one, so it is unique.

\nextprob
\hproblem Problem 3-12 a-c
Suppose $X$ is a topological space and $(X_{\alpha})_{\alpha\in A}$ is an
indexed family of topological spaces.
\begin{subproblems}
\item For any subset $S\subseteq X$, show that the subspace topology on $S$ is
the coarsest topology for which $\iota_S:S\hookrightarrow X$ is continuous.

\item Show that the product topology is the coarsest topology on
$\prod_{\alpha\in A}X_\alpha$ for which every canonical projection
$\pi_\alpha:\prod_{\alpha\in A}X_\alpha\rightarrow X_\alpha$ is continuous.
		
\item Show that the disjoint union topology is the finest topology on
$\coprod_\alpha X_\alpha$ for which every canonical injection
$\iota_\alpha:X_\alpha\rightarrow\coprod_\alpha X_\alpha$ is continuous.
\end{subproblems}

\subsol
Let $\tau_S$ be the subspace topology on $S$.  For the inclusion map $\iota_S$ to
be continuous, we must have that $\iota_S^{-1}(U)$ is open in $S$ whenever
$U$ is open in $X$.  But $\iota_S^{-1}(U)=U\cap S$, so the requirement is
that $U\cap S$ is open in $S$ whenever $U$ is open in $X$.

So, if $\tau$ is any topology on $S$ for which $\iota_S$ is continuous we must
have $U\cap S\in\tau$.  Since $U\cap S\in\tau_S$, we see that
$\tau_S\subseteq\tau$, i.e., $\tau_S$ is coarser than $\tau$.

Hence $\tau_S$ is the coarsest.

\subsol
Let $X=\prod_{\alpha\in A}X_\alpha$, and let $\tau_P$ be
the product topology on $X$.

For the projection map $\pi_\alpha$ to be continuous, we must have
that $V=\pi_\alpha^{-1}(U)$ is open in $X$ whenever $U$ is open in
$X_\alpha$.  Observe that
$V=\prod_{\beta\in X\setminus\alpha} X_\beta\times U$
which is open in $X$ under the product topology. Thus any topology $\tau$ for
which $\pi_\alpha$ is continuous must contain $V\in\tau_P$, so
$\tau_P\subseteq\tau$, i.e., $\tau_P$ is coarser than $\tau$.

Hence $\tau_P$ is coarsest.

\subsol
Let $X=\coprod_\alpha X_\alpha$, and
let $\tau_D$ be the disjoint union topology on $X$.  For the
canonical injection map $\iota_\alpha$ to be continuous, we require
$\iota^{-1}_\alpha(U)$ be open in $X_\alpha$ if $U$ is open in $X$.

But $\iota^{-1}_\alpha(U)=U\cap X_\alpha$, and this set being open
in $X_\alpha$
is precisely the condition on $U$ in $X$ for membership in $\tau_D$.
Hence if a
topology $\tau$ on $X$ contains a set $U\not\in\tau_D$, the corresponding
set in $X_\alpha$ will
not be open in $X_\alpha$ and then $\iota$ would not be continuous.
Thus for $\tau$ to result in continuous $\iota$ we must have that
$\tau\subseteq\tau_D$, i.e., $\tau_D$ is finer than $\tau$.

Hence $\tau_D$ is finest.

\end{aproblems}
\end{document}

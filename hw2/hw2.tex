\documentclass{homework}

% Include any special packages you might use.  Uncomment the
% following to use Times as the default font instead of
% TeX's default font of Computer Modern.
\usepackage{times,txfonts}
\usepackage{amsmath}
\usepackage{tensor}
\usepackage{tikz}
\usetikzlibrary{arrows}

% The following commands set up the material that appears
% in the header.
\doclabel{Math 651: Homework 2}
\docauthor{Kirk Hogenson}
\docdate{January 28, 2015}

\setlength\parindent{0pt}
\setlength\parskip{12pt}

% I've provided a file (hwextras.tex) with some commonly used extra 
% commands. If you've downloaded it, you can include it in your document
% by uncommenting the line below.  Feel free to make changes to that file.
\input{hwextras.tex}

\newcommand{\vv}{\mathbf{v}}
\newcommand{\calB}{\mathcal{B}}
\newcommand{\calF}{\mathcal{F}}
\newcommand{\calS}{\mathcal{S}}
\newcommand{\calU}{\mathcal{U}}
\DeclareMathOperator{\Int}{\mathrm{Int}}
\DeclareMathOperator{\Ext}{\mathrm{Ext}}
\newcommand{\ra}{\ensuremath{\rightarrow}}

\begin{document}

\begin{exercise}{1}
Suppose $\calB_1$ and $\calB_2$ are bases for topologies $\tau_1$
and $\tau_2$.  Show that $\tau_1\subseteq \tau_2$ if and only if
for every $B_1\in \calB_1$ and every $x\in B_1$ there is a $B_2\in \calB_2$
such that $x\in B_2\subseteq B_1$.
\end{exercise}
\solution
We'll first show the forward implication, so suppose that $\tau_1\subseteq\tau_2$.

Let $B_1\in\calB_1$ and $x\in B_1$.  Since $B_1\in\calB_1$, $B_1\in\tau_1$.
And since $\tau_1\subseteq\tau_2$, we see that
$B_1\in\tau_2$.
$\calB_2$ is a basis for $\tau_2$, therefore there exists $B_2\in\calB_2$ for
which $x\in B_2\subseteq U$ for any $U\in\tau_2$.  But $B_1\in\tau_2$, so there
exists $B_2\in\calB_2$ with $x\in B_2\subseteq B_1$ as required.

Now we show the reverse implication.  Let $U\in\tau_1$.

To show that $U\in\tau_2$, we have to show that for any $x\in U$ there exists
a $B_2\in\calB_2$ containing $x$ where $B_2\subseteq U$.

So take any $x\in U$.  Since $U\in\tau_1$ there exists $B_1\in\calB_1$ with
$x\in B_1\subseteq U$.  By assumption, since we have $B_1\in\calB_1$
and $x\in B_1$, there exists $B_2\in\calB_2$ such that $x\in B_2\subseteq B_1$.
And, since $B_1\subseteq U$, this implies $x\in B_2\subseteq U$, which
establishes that $U\in\tau_2$.

Hence $\tau_1\subseteq\tau_2$.

\hrulefill
\begin{exercise}{2}
Given a family $\{\tau_\alpha\}_{\alpha\in I}$ of topologies in $X$,
show that there is a unique smallest topology containing each $\tau_\alpha$.
Show also that there is a unique largest topology contained
in each $\tau_\alpha$.
Take advantage of past work!
\end{exercise}
\solution
Let $\calS=\bigcup_{\alpha\in I} \tau_\alpha$.  This forms a set which we
will show is a subbasis for some topology $\tau$.  $\tau$ will contain
all of the $\tau_\alpha$ by definition,
and because each $\tau_\alpha$ contains $X$,
$\calS$ contains $X$, too, and therefore $\calS$ covers $X$.  So, $\calS$
is a subbasis and hence generates a topology $\tau$ on $X$.

Let $\calB$ be the basis generated by $\calS$, and let $\tau_2$ be any
topology that contains each $\tau_\alpha$. We will show that
$\tau\subseteq\tau_2$.

Recall that $\tau\subseteq\tau_2$ if $\calB\subseteq\tau_2$.  But
$\calB\subseteq\tau_2$ if $\calS\subseteq\tau_2$, and $\calS\subseteq\tau_2$
by definition.

Therefore $\tau\subseteq\tau_2$, which means that $\tau$ is the smallest
topology containing each $\tau_\alpha$.

Now suppose $\tau'$ is another smallest topology.  We have just shown
that $\tau'\subseteq\tau_2$ for any topology $\tau_2$ that contains each
$\tau_\alpha$; but $\tau$ is such a topology, so $\tau'\subseteq\tau$.
Similarly, $\tau\subseteq\tau'$, and so $\tau=\tau'$ and therefore $\tau$ is
unique.

Now we consider the largest topology contained in each $\tau_\alpha$.
Let $\tau=\bigcap_{\alpha\in I}\tau_\alpha$.  By definition $\tau$ is contained
in each $\tau_\alpha$, and we showed in the last homework that $\tau$
is a topology.

Suppose $\tau_2$ is another topology contained in every $\tau_\alpha$. Then
for any $U\in\tau_2$, for every $\alpha\in I$ we have $U\in\tau_\alpha$, and
thus $U\in\cap_{\alpha\in I}\tau_\alpha=\tau$.  So, $\tau_2\subseteq\tau$.

Therefore $\tau$ is the largest topology contained in every $\tau_\alpha$.
Uniqueness follows as before; if $\tau'$ is also a largest topology, then
$\tau'\subseteq\tau$ and $\tau\subseteq\tau'$, and hence $\tau=\tau'$.

\hrulefill
\begin{exercise}{3}
Let $\calB=\{[a,b):a,b\in \Rats\}$.  Show that $\calB$ satisfies
the covering and refinement properties of the Topology Construction Proposition
and hence generates a topology $\tau_\calB$.  Compare this topology to the
lower-limit topology $\tau_\ell$.  In particular, determine if it is finer or coarser
or neither or both.
\end{exercise}
\solution
Take any $x\in\RR$.  There exist $a,b\in\Rats$ with
$a<x<b$, so if we take $B=[a,b)$ then $x\in B$ and $B\in\calB$,
and thus $\calB$ covers $\RR$.

Now to show that $\calB$ satisfies the refinement property, we will
show that it is closed under non-empty intersection.  So take
$B_1=[a,b)$ and $B_2=[c,d)$ in $\calB$.  We want non-empty $B_1$
and $B_2$, so we must have $a<b$ and $c<d$.  Without loss of
generality, we may assume $a \le c$.

If $b\le c$ then the intersection will be empty, so take $c<b$.

Putting it all together, we have $a\le c<b$, and $c<d$, so there are two
cases for $d$: (a) $c<d<b$ and (b) $c<b<d$.  In case (a) we have
$B_1\cap B_2=[c,d) \in\calB$ and in case (b) we have
$B_1\cap B_2=[c,b) \in\calB$.  So either way the intersection is
in $\calB$ and therefore $\calB$ is closed under nonempty
intersection and hence satisfies the refinement property.

Since $\calB$ satisfies the covering and refinement properties,
according to the Topology Construction Proposition it generates
a topology, $\tau_\calB$.

$\tau_\calB$ is coarser than $\tau_\ell$, since $\tau_\calB \subset \tau_\ell$.
To establish this, we must show that $\tau_\calB$ is contained in $\tau_\ell$,
and that there is at least one element of $\tau_\ell$ not in $\tau_\calB$.

That $\tau_\calB \subseteq \tau_\ell$ follows from the result of Problem 1.
Let $B_1=[a,b)\in\calB$ and $x\in B_1$, i.e., $x\in[a,b)$.  Since the real 
numbers are dense, we can find a real number $c$ where $a<c<x$ and
a real number $d$ where $x<d<b$.  Now let
$B_2=[c,d)$.  Then $B_2$ is a basis set for the lower limit topology, and
$x\in B_2$ and $B_2\subset B_1$.  Therefore
$\tau_\calB \subseteq \tau_\ell$.

But the containment is strict: $\tau_\calB \subset \tau_\ell$ because $[e,\pi)\in\tau_\ell$
but it is not in $\tau_\calB$.
To show this, we will show that there is no $B\in\calB$ for
which $e\in B\subset[e,\pi)$, yet we have $e\in[e,\pi)$.

Suppose $e\in B$ with $B=[a,b)$ for $a,b\in\Rats$.
Then we must have $a\le e<b$.
But $a$ is rational so we cannot have $a=e$, thus $a<e$.
This means there is a number $c$ satisfying $a<c<e$, and therefore
$a<c<b$ which implies $c\in[a,b)=B$.  But $c<e$ so $c\not\in[e,\pi)$
and hence $B\not\subset [e,\pi)$, which means
$[e,\pi)\not\in\tau_\calB$.

\hrulefill
\begin{exercise}{4}
Let $\calF$ be a family of subsets of a set $X$.  Show that there is
a unique smallest topology containing $\calF$.

If $\calS$ is a subbasis in $X$, show that the topology generated by
$\calS$ is the smallest topology containing $\calS$. (In particular,
the topology generated by a basis is the smallest topology containing
the basis.)
\end{exercise}
\solution
We'll do the second part first.

Let $\tau_\calS$ be the topology generated by $\calS$, and $\tau$ be another
topology containing $\calS$.  Since $\calS\subseteq\tau$, we have
$\tau_\calS \subseteq \tau$.  So, $\tau_\calS$ is smaller than (or
equal to) $\tau$.

Note that $\tau_\calS$ is the unique smallest topology by the same argument as
that in Problem 2.  If $\tau'$ is another smallest topology containing $\calS$,
then we immediately have both $\tau\subseteq\tau'$ and $\tau'\subseteq\tau$,
i.e, that $\tau=\tau'$.

Now consider $\calF$, a family of subsets of $X$.

If $\calF$ covers $X$, then it is a subbasis for $X$ and hence generates a
topology.  If $\calF$ does not cover $X$, then the set $\calS=\calF\cup\{X\}$
does cover $X$ and is therefore a subbasis and hence generates a topology.
Let this topology be $\tau$.

The basis $\calB$ for $\tau$ is obtained from the subbasis by taking finite
intersections of sets from $\calS$.  Let $S_1,\ldots,S_n$ be sets from $\calS$.
Since $\calS=\calF\cup\{X\}$, each $S_i$ is either an element of $\calF$ or
$X$.  Since any intersection of a set with $X$ is the original set, any finite
intersection of sets from $\calS$ is also a finite intersection of sets
from $\calF$, except for the case where every $S_i$ is $X$.  Therefore, the
set $\calB_\calS$ (the set generated by taking all possible finite intersections
of sets in $\calS$) is related to $\calB_\calF$ (the set generated by taking all
possible finite intersections of sets in $\calF$) by
$\calB_\calS=\calB_\calF\cup\{X\}$.

The next step in generating the topology is to take unions of sets in the basis.
Since any union of a set with $X$ is $X$, the union of any collection of sets
containing $X$ is $X$, too.  Therefore, adding $X$ to the basis can only add
one additional set to the topology, $X$.  But if $X$ were not already in the
topology, then it wouldn't have been a topology at all.

So, by construction, the topology generated from $\calS$ is identical to
the topology generated from $\calF$, if it exists.

Now we can invoke the result of the second part of the problem to say that
the topology generated by the subbasis $\calS$ is the unique smallest topology
containing $\calF$.

\hrulefill
\begin{exercise}{5}
Let $X$ and $Y$ be metric spaces.  Show that $f:X\ra Y$ is continuous
if and only if $f^{-1}(U)$ is open in $X$ whenever $U$ is open in $Y$.
\end{exercise}
\solution
$(\implies)$:\\
Suppose that $f$ is continuous.
If we use the $\epsilon-\delta$ formulation of continuity in a metric
space, this means that for a given $\epsilon>0$ and an $x\in X$, there
exists a $\delta>0$ such that $x'\in B_\delta(x)$ implies
$f(x')\in B_\epsilon(f(x))$.

Now we must show that when $U$ is open in $Y$, $f^{-1}(U)$ is open in $X$.
So, take any open set $U$ in $Y$, and any $y\in U$.  Since $U$ is open
there exists $\epsilon>0$ for which $B_\epsilon(y)\subseteq U$.  If we
choose an $x\in X$ for which $f(x)=y$, then the continuity of $f$ means
there exists $\delta>0$ such that $x'\in B_\delta(x)$ implies
$f(x')\in B_\epsilon(y)\subseteq U$.  And since $f(x')\in U$, we must have
$x'\in f^{-1}(U)$.

$x'$ was any point in $B_\delta(x)$, hence $B_\delta(x)$
is an open ball around $x$
in $f^{-1}(U)$, meaning that $f^{-1}(U)$ is open in $X$.

$(\impliedby)$:\\
We must show $f$ is continuous,
assuming that $U$ open in $Y$ implies $f^{-1}(U)$ is open in $X$.  So,
let $x\in X$ and $f(x)\in Y$, and let $B_\epsilon(y)$ be an open
ball around $y$ in $Y$.

Since $B_\epsilon(y)$ is an open set, $f^{-1}(B_\epsilon(y))$ is
open as well.  So, we can choose $\delta>0$ so that $B_\delta(x)$
is an open ball of radius $\delta$
around $x$ with $B_\delta(x)\subseteq f^{-1}(B_\epsilon(y))$.

Therefore for any $x'\in B_\delta(x)$, we have $x'\in f^{-1}(B_\epsilon(y))$,
and so $f(x')\in B_\epsilon(y)$, and hence $f$ is
continuous.

\hrulefill
\begin{exercise}{6}
 Let $A$ be a subset of a topological space $X$, and let $\calB$ be
a basis for the topology.
\begin{itemize}
\item[(a)] Show that $x\in\overline{A}$ if and only if for
every $B\in\calB$ with $x\in B$, $B\cap A\neq\emptyset$.
\item[(b)] Show that $x\in\partial A$ if and only if for every $B\in\calB$
with $x\in B$, $B\cap A\neq\emptyset$
and $B\cap A^c\neq \emptyset$
\item[(c)] Show that $\Int(A)\cap\partial A = \emptyset$ and
$\overline{A}=\Int(A)\cup \partial A$.
\end{itemize}
You are not allowed to cite Proposition 2.8 to prove these results.
\end{exercise}
\solution

\textbf{Lemma:}\\
We need a few preliminary results from exercises given in class.

First, we will need to use the fact that $\overline{A}$ is closed.
This follows from the fact that
$\overline{A}$ is an intersection of closed sets.

Secondly, we will need to know that $A\subseteq\overline{A}$.  This also follows
immediately from the definition of closure; $\overline{A}$ is an intersection
of sets containing $A$ and so will contain $A$ as well.

Thirdly, we will make use of the fact that $\Int(A)\subseteq A$, and that
$\Int(A)$ is an open set.  This also
follows directly from the definition: $\Int(A)$ is the union of all open sets
in $A$, so $x\in\Int(A)\implies x\in A$.  Moreover, the union of any collection
of open sets is open.

Finally, we will need to use the text's notion of the boundary,
$\partial A = X \setminus (\Int A \cup \Ext A)$.  We proved this in class,
except for the step $\Ext A = (\overline{A})^c$.  So, we will prove that
now.

By definition, $\Ext A$ is $\Int(A^c)$, and the interior of a set is the
union of all open sets contained in the set, in this case $A^c$.  If
we let $\calU_{A^c}$ be this collection of sets, then
$\calU_{A^c} = \{U\in\tau:U\subseteq A^c\}$.  Then we have
$$ \Int(A^c) = \bigcup_{U\in\calU_{A^c}} U. $$

The closure is the intersection of all closed sets containing
a set.  If $V$ is a closed set containing $A$, $V^c$ is an open set,
and it does not intersect $A$, i.e., it is in $A^c$.  So, the
closure is equivalent to the intersection of all sets $V$ such
that $V^c$ is open and $V^c\subseteq A^c$, i.e., $V^c\in\calU_{A^c}$.
So, we can write
$$ \overline{A} = \bigcap_{U\in\calU_{A^c}} U^c $$
and applying De Morgan's Law to find $\overline{A}^c$:
\begin{align*}
\overline{A}^c &= \left(\bigcap_{U\in\calU_{A^c}} U^c \right)^c \\
 &= \bigcup_{U\in\calU_{A^c}} (U^c)^c \\
 &= \bigcup_{U\in\calU_{A^c}} U \\
\end{align*}
Therefore $\Ext(A)=(\overline{A})^c$, and this completes the
proof that
$\partial A = X \setminus (\Int A \cup \Ext A)$.

\textbf{part (a)}\\
First, we assume $x\in\overline{A}$ and show that for every $B\in\calB$
with $x\in B$, we will have $B\cap A\ne\emptyset$.

Let $x\in\overline{A}$.  Then by definition $x$ is in every closed set that 
contains $A$.  Now take a $B\in\calB$ with $x\in B$.
If $B\cap A=\emptyset$, then $A\subseteq B^c$.  But $B^c$ is a closed set,
and it contains $A$, so $x\in B^c$.  This is a contradiction since $x\in B$,
thus $B\cap A$ is not empty.

Now suppose conversely that for every $B\in\calB$ with $x\in B$ we have
$B\cap A\ne\emptyset$.

Suppose $x\not\in\overline{A}$, then $x\in \overline{A}^c$. Since $\overline{A}$
is closed, $\overline{A}^c$ is open.
So, there exists $B\in\calB$ with $x\in B$ and $B\subseteq\overline{A}^c$.  This
implies that $B\cap\overline{A}=\emptyset$, which then means
that $B\cap A=\emptyset$ since
$A\subseteq \overline{A}$.  But, $B$ was supposed to satisfy $B\cap A\ne\emptyset$,
hence $x\in\overline{A}$.

\textbf{part (b)}\\
$(\implies)$:\\
Let $x\in\partial A$.  Then $x\in\overline{A}\cap\overline{A^c}$, i.e., $x$ is
in both $\overline{A}$ and $\overline{A^c}$.

Since $x\in\overline{A}$, from part (a) we have that for every $B\in\calB$ with
$x\in B$, $B\cap A\ne\emptyset$.

Similarly, since $x\in\overline{A^c}$, from part (a) we have that for every $B\in\calB$ with
$x\in B$, $B\cap A^c\ne\emptyset$.

$(\impliedby)$:\\
According to part (a), if we have that for every $B\in\calB$ with $x\in B$ and
$B\cap A\ne\emptyset$, then $x\in\overline{A}$.

Similarly, applying the same reasoning to $A^c$, since we have $B\cap A^c\ne\emptyset$, too,
we see that $x\in\overline{A^c}$.

Therefore $x\in\overline{A}$ and $x\in\overline{A^c}$, hence $x\in\partial A$.

\textbf{part (c)}\\
First, we'll show that $\Int(A) \cap \partial A = \emptyset$.  According to the
Lemma, we have that $(\partial A)^c = \Int (A) \cup \Ext(A)$.

If $x\in\Int(A)$, then
$x\in\Int(A)\cup\Ext(A)=(\partial A)^c \implies x\not\in \partial A$.\\
If $x\in \partial A$, then $x\not\in\Int(A)\cup\Ext(A) \implies x\not\in\Int A$.

So, $\partial A \cap \Int(A) = \emptyset$.

Next we show that $\overline{A} = \Int(A) \cup \partial A$.

Take any $x\in \Int(A)\cup\partial A$.  Then in $x\in\Int(A)$ or
$x\in\partial A$.  If $x\in\Int(A)$, then $x\in\overline{A}$, because
$\Int(A)\subseteq A\subseteq \overline{A}$.  If $x\in\partial A$ then
$x\in \overline{A}\cap\overline{A^c}$ and thus $x\in \overline{A}$.
So in either case $x\in\overline{A}$, therefore
$\Int(A)\cup\partial A\subseteq\overline{A}$.

Now take $x\in\overline{A}$.  So, from (a), every $B\in\calB$ with $x\in B$
we will have $B\cap A\ne\emptyset$.

Since $\partial A = X \setminus (\Int(A) \cup \Ext(A))$, any point $x$ in all
of $X$ must be in one of $\Int(A)$, $\Ext(A)$, or $\partial A$.

We already have that $x\in\overline{A}$.  Consider the possibility that also
$x\in\Ext(A)$.  Then $x\in\Int(A^c)$, which is an open set. Hence
there exists $B\in\calB$ with $x\in B$ with $B\subseteq \Int(A^c)\subseteq A^c$,
which means $B\cap A=\emptyset$.  But according to the result of part (a), 
$x\in\overline{A}$ implies that there are no such sets $B$.  Hence $x\not\in\Ext(A)$.

So it must be the case that $x\in\Int(A)$ or $x\in\partial A$,
i.e., that $x\in\Int(A)\cup\partial A$.  We started with $x\in\overline{A}$, therefore
$\overline{A}\subseteq\Int(A)\cup\partial A$.

Thus $\overline{A} = \Int(A)\cup \partial A$.

\end{document}

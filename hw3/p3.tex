\documentclass{homework}

% Include any special packages you might use.  Uncomment the
% following to use Times as the default font instead of
% TeX's default font of Computer Modern.
\usepackage{times,txfonts}
\usepackage{amsmath}
\usepackage{multicol}

% The following commands set up the material that appears
% in the header.
\doclabel{Math 651: Homework 3, Problem \#3 Redo}
\docauthor{Kirk Hogenson}
\docdate{}

\setlength\parindent{0pt}
\setlength\parskip{12pt}
\setlength\textheight{10in}

% I've provided a file (hwextras.tex) with some commonly used extra 
% commands. If you've downloaded it, you can include it in your document
% by uncommenting the line below.  Feel free to make changes to that file.
\input{hwextras.tex}

\newcommand{\vv}{\mathbf{v}}
\newcommand{\ra}{\rightarrow}
\newcommand{\calA}{\mathcal{A}}
\newcommand{\calB}{\mathcal{B}}
\DeclareMathOperator{\Int}{\mathrm{Int}}
\begin{document}
\begin{exercise}{3. Lee, Problem 2-4}
Let $X$ be a topological space and let $\calA$ be a collection of subsets
of $X$.  Prove the following containments:
\vspace{-.1in}
\begin{multicols}{2}
\begin{itemize}
\item[(a)] $\displaystyle\overline{\bigcap_{A\in\calA} A} \subseteq \bigcap_{A\in\calA} \overline{A}$.
\item[(b)] $\displaystyle\overline{\bigcup_{A\in\calA} A} \supseteq \bigcup_{A\in\calA} \overline{A}$.
\item[(c)] $\displaystyle\Int\left(\bigcap_{A\in\calA} A\right) \subseteq \bigcap_{A\in\calA}\Int A$.
\item[(d)] $\displaystyle\Int\left(\bigcup_{A\in\calA} A\right) \supseteq \bigcup_{A\in\calA}\Int A$.
\end{itemize}
\end{multicols}
\vspace{-.1in}
When $\calA$ is a finite collection, show that equality holds in (b) and (c), but
not necessarily in (a) or (d).
\end{exercise}
\vspace{-.2in}
\solution
\textbf{Lemma:} \textit{$A\subseteq B\implies\overline{A}\subseteq\overline{B}$}\\
For any set $B$ we showed on the last homework that $B\subseteq\overline{B}$. So
if $A\subseteq B$, then $A\subseteq\overline{B}$.  By definition $\overline{A}$ is
contained in all closed sets containing $A$, and we've just shown that $\overline{B}$
is one such set.  Hence $\overline{A}\subseteq\overline{B}$.

\textbf{part (a)}\\
Let $x\in \overline{\bigcap_{A\in\calA} A}$, and $U$ be an open set with $x\in U$.
Then $U\cap \bigcap_{A\in\calA} A\ne\emptyset$.  So, for each $A$ we have
$U\cap A\ne\emptyset$.  Thus for each $A$, $x\in\overline{A}$, hence
$x\in\bigcap_{A\in\calA}\overline{A}$.

The reverse containment (and hence equality) does not hold even for finite $\calA$, e.g.,
in $\RR$ take $A_1=(0,1)$ and $A_2=(1,2)$, then
$\overline{A_1\cap A_2}=\emptyset$, however $\overline{A_1}\cap\overline{A_2}=\{1\}$.

\textbf{part (b)}\\
Take $x\in\bigcup_{A\in\calA} \overline{A}$, so
for some $A_0\in\calA$, $x\in\overline{A_0}$.
Since $A_0\subseteq \bigcup_{A\in\calA} A$, from the Lemma we have
$\overline{A_0}\subseteq \overline{\bigcup_{A\in\calA} A}$.  Hence
$x\in\overline{\bigcup_{A\in\calA} A}$.

Now let $\calA$ be finite, and take $x\in\overline{\bigcup_{A\in\calA} A}$,
and any open $U$ with $x\in U$.  Then we have
$U\cap \bigcup_{A\in\calA} A\ne\emptyset \implies
\bigcup_{A\in\calA} (U\cap A)\ne\emptyset$.  So for some $A_0\in\calA$, $U\cap A_0\ne\emptyset$.
Because $U$ was any open set containing $x$, we conclude $x\in\overline{A_0}$, so $x\in\bigcup_{A\in\calA} \overline{A}$.

\textbf{part (c)}\\
Take $x\in\Int\bigcap_{A\in\calA} A$.  Then there exists an open set $U$ containing
$x$ such that $U\subseteq \bigcap_{A\in\calA} A$.  So $U\subseteq A$ for each $A\in\calA$,
which means $x\in\Int A$ for each $A\in\calA$.  Hence $x\in\bigcap_{A\in\calA}\Int A$.

Now suppose $\calA$ is finite.
Take $x\in\bigcap_{A\in\calA}\Int A$.  Then $x\in\Int A$ for every $A\in\calA$. 
For each $A\in\calA$, let $U_A$ be an open set such that $U_A\subseteq A$,
and $x\in U_A$.  Define $U=\bigcap_{A\in\calA} U_A$.
Note that $x\in U$, and since $\calA$ is finite, $U$ is open.
Also, $U\subseteq A$ for
each $A\in\calA$, so $U\subseteq\bigcap_{A\in\calA} A$.
Hence $x\in\Int\bigcap_{A\in\calA} A$.

\textbf{part (d)}\\
Take $x\in\bigcup_{A\in\calA}\Int A$.  Then for some $A_0\in\calA$, we will
have $x\in\Int A_0$, so let $U$ be an open set with $x\in U\subseteq A_0$.
But $A_0\subseteq \bigcup_{A\in\calA} A$, so $U\subseteq \bigcup_{A\in\calA} A$,
and hence $x\in\Int \bigcup_{A\in\calA} A$.

Equality does not hold even for finite $\calA$, e.g., in $\RR$
take $A_1=(0,1]$ and $A_2=(1,2)$. Then $\Int(A_1\cup A_2)=(0,2)$ and
$\Int(A_1)\cup\Int(A_2)=(0,1)\cup(1,2)$.
\end{document}
